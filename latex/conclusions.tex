\section{Conclusions}
\label{sec:conclusions}

This note presents measurements of the hit rates in the MDTs and CSCs of the ATLAS Muon Spectrometer in 2015 operation. The hit rates are considered as a function of detector region, instantaneous luminosity, and transverse distance from the beampipe. The hit rates show linear behavior as a function of the instantaneous luminosity, indicating good performance.

Additionally, the hit rates are projected to future data-taking conditions by extrapolating the linear behavior to the expected conditions in Run 3 and the HL-LHC. These runs are expected to operate with more filled bunches and higher instantaneous luminosity than are reached in 2015 operation. The maximum projected hit rate in the small wheel at HL-LHC conditions is 11 $\text{kHz} / \text{cm}^2$, though this projection does not consider future changes to the ATLAS shielding or beampipe.

