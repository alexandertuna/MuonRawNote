\section{Dataset}
\label{sec:dataset}

The data in this note are taken from proton-proton collisions provided by the LHC in 2015. The collision energy increased from 8 TeV to 13 TeV and the spacing of proton bunches decreased from 50 ns to 25 ns relative to the last data-taking campaign, which ended in 2012.

Collision events are selected if they pass a so-called ``zero bias'' trigger to study a generic sample of proton-proton collisions. The zero bias trigger fires one LHC orbit after a EM15 trigger fires, thus providing an unbiased and luminosity-dependent sample of collisions. The trigger is prescaled to have a rate of 5 Hz, which gives hundreds of thousands of unbiased collision events for a typical LHC fill lasting 10-20 hours.

Data are taken from many runs in 2015. The instantaneous luminosity delivered by the LHC increased throughout the year and reached a maximum of $\sim\!5\times10^{33} \text{cm}^{-2} \text{s}^{-1}$. This is smaller than the maximum instantaneous luminosity reached in 2012, $\sim\!8\times10^{33} \text{cm}^{-2} \text{s}^{-1}$, mostly due to the larger $\beta^\ast$ of the proton beams in 2015 relative to 2012. Most of the runs considered in this note last many hours so that the hit rates can be smoothly measured as a function of the instantaneous luminosity.

The number of filled bunches in the LHC proton beams changes throughout 2015 as the machine is commissioned. The MDTs are especially sensitive to this structure because they have a long livetime of 1300 ns (52 BC) and are therefore susceptible to particle fluxes long before and after the collision of interest. For comparison, the TGCs have a livetime of 50-75 ns.

Data are only included if all ATLAS detectors and magnets are operational and fit for physics analysis.



