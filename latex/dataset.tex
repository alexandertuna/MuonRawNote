\section{Dataset}
\label{sec:dataset}

The data in this note are taken from proton-proton collisions provided by the LHC and recorded by the ATLAS detector in 2015. The center-of-mass collision energy is 13 TeV and the spacing of proton bunches is 25 ns. This is an increase of energy and decrease of bunch spacing relative to the data-taking campaign in 2012, which operated at 8 TeV and 50 ns, respectively. The data are required to be collected during stable beam conditions, and all ATLAS sub-systems are required to be operational and fit for physics analysis.

Collision events are selected if they pass a so-called ``zero bias'' trigger to study a generic sample of proton-proton collisions. The zero bias trigger fires one LHC orbit after only a modest deposit of energy is observed in the electromagnetic calorimeter (\texttt{EM15}), thus providing an unbiased and luminosity-dependent sample of collisions. Events firing this selection are randomly recorded such that the rate of the trigger is 5 Hz, which gives hundreds of thousands of unbiased collision events for a typical LHC fill lasting 10-20 hours.

Data are taken from many days of operation in 2015, which provides a broad sample of collision conditions. The instantaneous luminosity delivered by the LHC increases throughout the year and reached a maximum of about $5\times10^{33} \text{ cm}^{-2} \text{ s}^{-1}$. This is smaller than the maximum instantaneous luminosity reached in 2012, about $8\times10^{33} \text{ cm}^{-2} \text{ s}^{-1}$, mostly due to the larger $\beta^\ast$ of the proton beams in 2015 relative to 2012. Most of the runs considered in this note last many hours so that the hit rates can be smoothly measured as a function of the instantaneous luminosity. The definition of the zero bias trigger remains unchanged throughout the year.

The number of filled bunches in the LHC proton beams also increases throughout 2015 as the machine is commissioned. The LHC can maximally hold about 2800 bunches of protons in the ring with 25 ns bunch spacing, including space for beam dumps, compared to half as many with 50 ns spacing. The earliest days of data-taking in 2015 have 10 or fewer colliding bunches, and the latest have about 2200. The full LHC ring is expected to be filled in 2016 data-taking and beyond.

The MDTs are especially sensitive to this structure because they have a long timing window of 1300 ns. Many hits recorded in the MDTs are therefore from incident particles arising from proton collisions long before and after the collision which triggered the event. For comparison, the TGCs have a livetime of 50-75 ns.

% Data are only included if all ATLAS detectors and magnets are operational and fit for physics analysis.



