\section{Data set}
\label{sec:data}

The data in this note are taken from proton-proton collisions provided by the LHC in 2015. The collision energy increased from 8 TeV to 13 TeV and the spacing of proton bunches decreased from 50 ns to 25 ns. 

The bunch structure of the LHC proton beams changed throughout 2015 as the machine was commissioned. The MDTs are especially sensitive to this structure because they have a long livetime of 1300 ns (52 BC) and are therefore susceptible to particle fluxes before and after the collision of interest. For comparison, the TGCs have a livetime of 50 ns.

Data are only included if all ATLAS detectors and magnets are operational and fit for physics analysis.

To study a generic sample of proton-proton collisions, events are selected if they pass a so-called ``zero bias'' trigger. The zero bias trigger fires one LHC orbit after a EM15 trigger fires, thus providing an unbiased and luminosity-dependent sample of collisions. The trigger is prescaled to have a rate of 5 Hz. 


