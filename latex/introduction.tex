\section{Introduction}
\label{sec:intro}

Muons are important at $\pt$ of a few GeV ($H\rightarrow ZZ^\star\rightarrow 4\ell$, $H\rightarrow J/\psi \gamma \rightarrow 2\ell\gamma$) and up to $\pt$ of many TeV ($W'$, $Z'$).

The MDTs had problems in 2012 and were upgraded during LS1 to address this. The CSCs had updates to the RODs during LS1. Understanding their performance in new data is important.

In the next long shutdown, projected to be in 2018, the current small wheel will be replaced by the so-called New Small Wheel (NSW). One of the driving performance requirements of the NSW is the ability to cope with a large flux of incident particles. Extrapolating the current particle flux to post-LS2 conditions is therefore important.

Section \ref{sec:detector} describes the ATLAS detector, and in particular, the MDTs and CSCs. Section \ref{sec:dataset} describes the data set used in this study. Section \ref{sec:hitrates} describes the measured hit rates in the MDTs and CSCs as a function of detector region, luminosity, distance from the beam pipe. Section \ref{sec:extrapolations} describes extrapolations of the hit rates from the current LHC conditions to Run 3 and the HL-LHC conditions. Section \ref{sec:conclusions} concludes with a summary of the studies.


