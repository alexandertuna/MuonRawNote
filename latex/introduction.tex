\section{Introduction}
\label{sec:intro}

The efficient detection of muons at the ATLAS experiment is crucial in tests of the Standard Model (SM) and searches for new physics beyond the SM. The ATLAS Muon Spectrometer (MS) is built to operate in a harsh radiation environment and detect muons with momenta as small as a few GeV and as large as many TeV. These muons are a prerequisite to measuring the characteristics of the Higgs boson via $H\rightarrow ZZ^\star\rightarrow 4\ell$ and for searching for new heavy particles which can decay to muons, such as $W'$ and $Z'$. 

The MS is composed of four independent detector technologies, two of which are for fast triggering (RPCs, TGCs) and two of which are for precision trajectory and momentum measurements (MDTs, CSCs). This note describes the measurement of hit rates in the MDTs and CSCs in 2015 proton collisions at the LHC. 

% The MDTs cover the entire solid angle of the ATLAS interaction point up to a pseudo-rapidity of 2.7, and the CSCs are deployed in the region of the endcap small wheel closest to the beampipe because of their capacity for withstanding large incident particle flux.

The purpose of reporting the hit rates in 2015 operation is two-fold. First, the ATLAS detector underwent many upgrades in the recent LHC Long Shutdown 1, and measuring the hit rates is an effective gauge of the detector performance as a function of incident particle flux. Second, in the next Long Shutdown 2, the current small wheel will be replaced by the so-called New Small Wheel (NSW), and one of the driving performance requirements of the NSW is the ability to cope with a large flux of incident particles. Hit rates from 2015 operation can be extrapolated to future conditions, and these can serve as input to MS upgrades like the NSW and beyond.

The note is organized as follows. Section \ref{sec:detector} describes the ATLAS detector, and in particular, the MDTs and CSCs. Section \ref{sec:dataset} describes the data set used in this study. Section \ref{sec:hitrates} describes the measured hit rates in the MDTs and CSCs as a function of detector region, luminosity, distance from the beam pipe. Section \ref{sec:extrapolations} describes extrapolations of the hit rates from the current LHC conditions to Run 3 and the HL-LHC conditions. Section \ref{sec:conclusions} concludes with a summary of the studies.


